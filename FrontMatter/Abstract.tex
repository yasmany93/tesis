\begin{abstract}

Los más grandes desastres naturales que recoge la historia de nuestro país han estado asociados a los ciclones tropicales. La gran actividad ciclónica ocurrida en los últimos años, ha centrado la atención sobre la climatología de estos, su variabilidad y su tendencia a largo plazo. Varios huracanes han ocasionado desastres de gran significación debidos, fundamentalmente, al número de personas que murieron como consecuencia del impacto de la tormenta. De acuerdo a la trayectoria revisada de huracanes que azotaron a la Isla de Cuba, se decidió hacer la actualización del software TkHURS para obtener los períodos de retorno y calcular las frecuencias estimadas a través del ajuste de un Modelo de Poisson a la variable que cuenta el número de huracanes por año que han azotado a Cuba, en el período de 1791-2018, a partir de la Cronología de los Ciclones Tropicales y los Estados Generales del Tiempo. Se actualizó la metodología permitiendo agregar las nuevas funciones del software para una mayor comprensión a la hora de su utilización. Se realizó la validación a través de la región Central en el período 1980-2018 con los huracanes intensos, donde se pudo observar que se puede esperar un huracán intenso cada 10 años. El gráfico de pastel muestra que esta región es más afectada por los de categoría SS4 con un 50\%. Además, permitirá realizar mejores investigaciones y brindar servicios a distintas instituciones para obtener beneficios económicos. 

\begin{description}
\item[Palabras claves:]{ Huracanes, Modelo de Poisson, Software, Períodos de Retorno, TkHURSv1.1}  
\end{description}

\end{abstract}