\chapter{Normalización de los Datos}
\label{chap:prep}

Un \emph{Tweet} es un mensaje de 140 caracteres donde es común encontrar gran uso de abreviaturas y jerga
de la web, por lo general para expresar opiniones se utiliza gran diversidad de emoticones y se deforma el
lenguaje ya que la ortografía y la gramática del idioma no son muy relevantes en estos mensajes. Es muy
usual que porten etiquetas y marcas propias de \emph{Twitter} como las de usuario, tópicos o \emph{reTweet}~(utilizado
para indicar que algún usuario replicó el mensaje) y que hagan referencia a diferentes sitios de la web.

\begin{verbatim}
bloggerscuba >Ya vieron el artículo de Global Voices? http://bit.ly/7uPumv

yudivian jajaja :D

RT yudivian: Vas a estar pa record roger213tm dos #nutella 
in one week &lt;- Por lo menos tremendo average :P

RT MaiteLpez: caroboris NUnca estare en el grupo de industriale 
sen Facebook ni en ningun lugar #VillaClara campeon &lt;- #EPD

RT: oxigeno: roger213tm por las #torticas!! &lt;- SSSSSufre... ajaja

nos perdimos de aki jijiji

&lt;(o-o)&gt;
\end{verbatim}

En la web los mensajes no están sujetos a estrictas reglas sintácticas del lenguaje \cite{Merlo2010} lo que dificulta la
aplicación de algoritmos de minería de opinión sobre ellos. Por tanto se propone un proceso de conversión
de los \emph{Tweets} a lenguaje natural. La primera diferencia de un \emph{Tweet} con un texto en lenguaje natural es
el uso indiscriminado de abreviaturas y jerga utilizada en la web debido a la restricción en la longitud de
140 caracteres.La jerga también llamada \emph{netspeak}~(uso informal de palabras y expresiones que no están
consideradas en el idioma o dialecto del hablante) ofrece como resultado más de una abreviatura para
una única palabra lo que en el entrenamiento provoca que el peso de la palabra quede repartido entre las
variaciones de la misma o en la clasificación una de las variaciones obtenga un peso mínimo aunque la
palabra esté entre los términos determinantes a la hora de clasificar. Por tanto es evidente la necesidad de
traducir la jerga a un lenguaje más convencional, para esto se propone el uso de diccionarios en diferentes
idiomas~(español e inglés) de la forma: jerga--idioma que se consultará para la conversión.

También es frecuente la utilización de emoticones o \emph{smileys} empleados para darle expresividad al
mensaje dado que internet es un medio frío y a menudo se desean expresar opiniones positivas o negativas.
A los emoticones se le da usos más complejos de detectar como es la ironía, de ahí que sean muy
importantes al realizar un análisis de sentimientos. Se dispone de un diccionario con 144 emoticones y
las frases que definen su semántica lo más breve posible.
Como parte del preprocesamiento se hace tratamiento para evitar las diferencias entre mayúsculas y
minúsculas que pueden suponer diferencias para el ordenador cuando realmente se hace referencia a la
misma palabra~(Ejemplo: Hola, hola, hOlA, HOLA). Este paso consiste simplemente en convertir todo
a minúsculas.

Una deformación común en internet es la modificación de la sintaxis de las palabras repitiendo
caracteres dentro de la misma con el objetivo de enfatizarla~(Ejemplo: "hooooooola"). Generalmente
la mayor o menor presencia de este efecto está muy ligado al tipo de contenido con que se trabaja y
dentro de este al estilo propio del autor. En las páginas de noticias escritas por periodistas normalmente
hay un mayor control de la ortografía mientras que las páginas de reseñas o en los blogs este fenómeno
es mucho más frecuente y en \emph{Twitter} su presencia se acentúa todavía más ya que en este lenguaje no
es relevante la ortografía. Para un análisis posterior del texto es necesario suprimir las letras repetidas,
pero existen palabras correctas con más de un carácter igual consecutivo~(Ejemplo: \emph{good}, acción). Por
lo tanto se propone suprimir los caracteres repetidos hasta dos ocurrencias. Existe otro problema y es
que no todas las palabras con 2 caracteres repetidos son transformadas y pueden haber sido escritas
incorrectamente por lo que llegado a este punto se utiliza un corrector ortográfico que determine las
palabras correctas. Si el corrector da más de una forma correcta para la misma palabra esta se deja como
está en el texto al igual que si no se encuentra ninguna forma correcta y se continua con el proceso de
análisis.

En \emph{Twitter} es relativamente común la utilización de referencias a otros sitios de la web con frecuencia
blogs, noticias o reseñas. La dirección de estos sitios, generalmente acortada usando un servicio específico
para ese fin, aparece en el propio mensaje. El problema surge cuando en la URL se encuentran palabras
que pueden influir sobre el clasificador de forma no deseada. Por esto es necesario eliminarlas ya que
forman parte del texto del mensaje
En un \emph{Tweet} se hace referencia no solo a sitios de internet, también pueden hacerse referencias a
otros usuarios de \emph{Twitter} a un tópico en especial o a otros \emph{Tweets} mediante el uso de marcas o etiquetas
propias de \emph{Twitter} que aparecen en el texto y pueden influir de la misma manera que las URL sobre los
clasificadores por lo tanto también son eliminadas.
Al finalizar este preprocesamiento se considera que el \emph{Tweet} se encuentra en lenguaje natural y se
respetan los signos de puntuación que serán utilizados en análisis posteriores.

%--------------------------------
Como se describe anteriormente el determinar el idioma del \emph{Tweet} permite cargar los diccionarios de
archivos de texto, según el idioma determinado. Aunque en una primera etapa solo se considerarán
mensajes en Español y posteriormente en Inglés. También se carga por defecto el diccionario de emoticones
que no depende del idioma del texto. Los diccionarios están almacenados en archivos de texto~(de la forma
jerga/emoticon : significado):

\begin{verbatim}

CU2MORO : ”See you tomorrow”

Googlear : Buscar en Google

:-] : positivo

’] :’ : negativo
\end{verbatim}

El procesamiento sobre el \emph{Tweet} es eliminar las etiquetas de \emph{Twitter} y las URLs mediante un autómata
finito determinista que elimina todos los caracteres detrás de un $\sharp$ o que se reconozca como
la dirección de un sitio. También se elimina la etiqueta ”RT” si se encuentra en cualquiera de sus
formas~(rt, Rt, rT) y si está rodeada de espacios ya que de lo contrario se pueden eliminar partes de
palabras correctas.
Para identificar los emoticones se propone el uso de expresiones regulares y el diccionario de emoticones
para crearlas. Estos son eliminados del texto para que no se confundan con la jerga o signos de
puntuación y evitar que cambien de sentido debido a otras transformaciones.
Para analizar el texto es necesario encontrar una unidad básica~(en nuestro caso utilizamos las palabras. 
Se entenderá por palabra inicialmente el resultado de dividir el \emph{Tweet} por espacios. Luego cada
palabra es verificada contra el diccionario de jerga después de convertirla a minúsculas y si no se considera
jerga se divide la palabra en los grupos continuos de puntuación y de caracteres alfabéticos donde
los últimos continúan siendo procesados como palabras nuevas.
Las nuevas palabras se vuelven a comparar con jerga y se les suprimen los caracteres repetidos hasta
dos ocurrencias para luego verificar la palabra correcta mediante los algoritmos de corrección ortográfica.

\section{Stemming}
Para el \emph{stemming} se propone una variante del algoritmo de Porter para Español de la 
librería nltk. Al algoritmo original se le hicieron varios cambios:
\begin{itemize}
 \item Considerar los sufijos: ito ita azo aza itos itas azos azas
 de los diminutivos y aumentativos.
 \item Se agrega la regla de que estos sufijos solo se podrán eliminar si lo que queda de la palabra es
 mayor o igual a tres caracteres.
 \item Se agrega la regla de que estos sufijos solo se podrán eliminar si al agregar la 
 última letra del sufijo al resto de la palabra, esta tiene sentido~(aparece en los diccionarios).
\end{itemize}

Ejemplo:

perr\emph{ito} 
perro (aparece en el diccionario)
--------------
perr

suscr\emph{ito}
suscro (no aparece en el diccionario)
------------
suscrit

\section{Stop Words}

Las \emph{stop words} son palabras que serán eliminadas una vez que sean encontradas en el texto.
Para esto se trabajo en un listado de palabras propuestas en diferentes sitios de internet 
como las \emph{stop words} más utilizadas, pero varias de estas palabras pueden aportar 
semántica a la opinión que se da en el mensaje. Por lo tanto se propone no incluir:

\begin{center}
\begin{tabular}{c c c}
  no & muy & ni \\
  si & contra & algunos \\
  pero & tanto & algunas \\
  más & mucho & poco \\ 
  como & nada & algo \\ 
  tanto & muchos & entonces \\
  cierto & bajo & verdad \\
  ciertos & arriba & verdadero \\
  ciertas & encima & verdadera \\
  cierta & valor & modo \\
  bien & mientras & \\
\end{tabular} 
\end{center}

