%cosas ya escritas

%TODO referencia a sitios de emoticones
  %wikipedia
  
  El trabajo con emoticones es más sencillo, puesto que existen varios sitios en internet
  que se dedican a crear bibliotecas de emoticones. Por lo tanto no es difícil crear
  un diccionario de emoticones bastante actualizado, ya que estos sitios generalmente
  agregan los nuevos emoticones que se utilizan en las redes sociales.
  
  %TODO referencia a sitios de jerga
  
  La jerga se encuentra en el mismo caso que los emoticones, con la diferencia
  de que los sitios le dan a cada palabra~(de la jerga) un significado usualmente 
  extenso. Esto no permite intercambiar a la palabra en el texto por su correspondiente 
  significado. Por lo tanto se requiere crear un diccionario que modifique el significado
  de cada elemento de la jerga por un significado sencillo que mantenga la idea esencial,
  sobretodo teniendo en cuenta la opinión que puede expresar. Por ejemplo palabras como 
  \emph{hacker} no aportan ninguna opinión por lo que no necesitan ser sustituidas en el texto.
  
  Los sitios que contienen jerga y emoticones no necesitan ser avalados de manera oficial
  puesto que su contenido solo se utilizará si es encontrado en el texto. Además el 
  lenguaje utilizado en las redes sociales no es correcto, ni oficial, sino producto 
  de lo que se usa en la red. Si los sitios contienen algún emoticón o jerga, es 
  bastante posible que lo encontremos en los mensajes de texto que se quieren analizar,
  ya que los usuarios son bastante susceptibles a utilizar los nuevos elementos
  que circulen en la red.
  
 %------------------------------------------------------------------
 
 \subsection{Criterios para etiquetar el corpus}
%revisados los mensajes 
%porque se determinaron solo esas clasificaciones
%que otras pueden tenerse en cuenta 
%que significa cada clasificación
%ejemplos
 
 A partir de una revisión general a los mensajes, se determinó
 clasificar solo en las categorías Objetivo, Subjetivo y esta última en Positivo, Negativo, Neutro.
 Una gran cantidad de categorías o contar con categorías similares hace más compleja la tarea
 de clasificación para los seres humanos. Por tanto se determinó utilizar pocas categorías, que resultaran 
 bien diferentes a los usuarios. Las categorías se especifican como: 
 
 \begin{description}
  \item[Objetivo:] Son mensajes que plasman un hecho, a menudo noticias.
  \item[Subjetivo:] Donde se expresa una opinión, se da un criterio.
  \item[Positivo:] Solo es aplicable a mensajes Subjetivos donde la opinión expresada es
  Positiva, se considera favorable por el autor.
  \item[Negativo:] Solo es aplicable a mensajes Subjetivos donde la opinión expresada es
  Negativa, se considera desfavorable por el autor.
  \item[Neutro:] Solo es aplicable a mensajes Subjetivos donde se expresa una opinión,
  pero esta no es negativa, ni positiva.
 \end{description}
 
%  Además se determinaron un conjunto de categorías, en que pueden clasificarse los mensajes aunque no fueron 
%  utilizadas en la clasificación final. A menudo estás categorías se solapan con otras y crean confusión. 
%  Aparecen en el texto fundamentalmente mezcladas con chistes, ironía y metáforas.
%  
%  \begin{description}
%   \item[Interrogativo:] Mensajes en los que se plantean preguntas.
%   \item[Irónico:] Mensajes donde se expresa una opinión contradictoria.
%   \item[Compuesto:] Mensajes donde aparece más de una clasificación, por ejemplo se expresan
%   dos opiniones diferentes.
%   \item[Reflexivo:] No se da una valoración directa de un tema, pero se expresa una opinión.
%   A menudo aparecen en forma de concejos o frases célebres.  
%  \end{description}

Para la creación del corpus se construyó una aplicación sobre la plataforma .Net donde 
los usuarios pudieran clasificar los
mensajes rápidamente. Los mensajes clasificados se almacenan en un archivo aparte y 
se eliminan del conjunto original. 
El mensaje a clasificar en cada momento se elige aleatoriamente. 
La aplicación se distribuyó entre 10 usuarios aproximadamente para que realizaran las tareas
de clasificación. 

%---------------------------------------------------------------------
Un primer paso en el análisis de opinión es determinar si el texto es objetivo o subjetivo.
A continuación se analiza la polaridad, es decir, separar las opiniones que expresan algo 
a favor, de las que expresan algo en contra. 

El trabajo de tesis se propone utilizar diferentes métodos de aprendizaje 
supervisado para la detección de opiniones en mensajes de \emph{Twitter},
así como el análisis de polaridad en los mismos. 

Un punto relevante del trabajo es la evaluación de los métodos de clasificación sobre
idioma español, puesto que la mayoría de los trabajos previos se concentran en el idioma inglés.
Esto conlleva a la necesidad de crear un corpus en idioma español. Para resolver 
este problema se propone un corpus de 2000 mensajes de \emph{Twitter} en español,
clasificados a mano.
%_--------------------------------------------------------------------

En \emph{Twitter} se denomina a los
mensajes \emph{tweets} que se muestran en la página del perfil de su autor y son enviados a
todos los usuarios ``suscritos'' al mismo, los denominados seguidores~(\emph{followers}). 

