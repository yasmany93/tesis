\chapter{Clasificación}
\label{chap:clas}

	La clasificación de los mensajes se divide en dos etapas:
	
	\begin{itemize}
	\item Determinar si el mensaje es objetivo o subjetivo.
	\item Clasificar los mensajes subjetivos en positivos o negativos.
	\end{itemize}
	
	Se propone la experimentación con los mismos clasificadores en las dos
	etapas, ya que ambas enfrentan el mismo problema teórico.

	El aprendizaje supervisado parece una alternativa natural para enfrentar este
	problema, ya que se desea asignar a cada mensaje una clasificación única.
	El conjunto de entrenamiento se construye a partir de la clasificación
	manual por parte de especialistas~(en este caso los autores) de una lista
	de \emph{tweets} extraídos del sitio de \emph{Twitter} empleando herramientas
	desarrolladas en la facultad para este propósito. Es necesario notar que 
	incluso la tarea de clasificación manual es difícil, debido a la ambigüedad
	inherente a estos mensajes. Por este motivo se ha escogido~(intentado confeccionar
	un conjunto de entrenamiento seleccionando) los mensajes que para un humano
	tienen una orientación bien definida, obviando los mensajes donde los clasificadores
	humanos estuvieron en desacuerdo.
	
	Los clasificadores empleados reciben un conjunto de entrenamiento en 
	la etapa de aprendizaje, que en este caso es obtenido a partir de un archivo de
	texto con una entrada por línea de la forma
	\verb|MENSAJE ====> CLASE|, donde la clase asignada es \verb|OBJETIVO|,
	\verb|NEGATIVO| o \verb|POSITIVO|.
	
	Existen varios algoritmos de clasificación, teniendo en cuenta el problema que se desea 
	resolver se eligieron para probar aquellos que son más recomendados en la literatura:
	\begin{itemize}
	    \item Naive Bayes.
	    \item Máquinas de soporte vectorial~(SVM).
	    \item Redes Neuronales.
	    \item K vecinos más cercanos.
	    \item Árboles de decisión.
	    \item Redes Bayesianas.
	    \item Máxima entropía.
	\end{itemize}
  
	La mayoría de los clasificadores utilizados forman parte de la herramienta \emph{sklearn}.
	La que cuenta con gran variedad de algoritmos de aprendizaje. Además contiene operaciones y estructuras
	de datos diseñadas para trabajar con vectores esparcidos y realizar validaciones sobre los resultados
	obtenidos. 
	

