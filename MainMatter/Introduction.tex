\begin{introduction}

Los más grandes desastres naturales que recoge la historia de nuestro país han estado asociados a los ciclones tropicales. Varios huracanes han ocasionado desastres de gran significación debidos, fundamentalmente al número de personas que murieron como consecuencia del impacto de la tormenta. La gran actividad ciclónica ocurrida en los últimos años, ha centrado aún más la atención sobre la climatología de los ciclones tropicales, su variabilidad y sus tendencias a largo plazo.\\ 
En el año 2000 se culminó un estudio cuyo objetivo era confeccionar una cronología de los ciclones tropicales que han afectado a Cuba, a la luz de los conocimientos actuales, que sirviera de base para actualizar los conocimientos sobre la climatología de los ciclones tropicales de Cuba, su variabilidad y los factores que la regulan. Posteriormente, se ejecutó un nuevo proyecto denominado “Climatología de los ciclones tropicales de Cuba”, el que estuvo dirigido a prolongar hacia el pasado los resultados antes alcanzados en la cronología \cite{DK4, DK5}. La misma al día de hoy abarca el periodo 1791 – 2018 para los huracanes.\\
Es necesario saber los períodos de retorno de la afectación de huracanes a cada una de las provincias, por región y en general a Cuba. También se tiene en cuenta los Resúmenes de la Temporada Ciclónica elaborados por el Centro de Pronósticos del Tiempo del Instituto de Meteorología para los años posteriores a 1998. Se tiene en cuenta la categorización con que cada huracán afectó a cada provincia, utilizando para ello la escala Saffir-Simpson. En la Tabla \ref{escala_saffir_simpson} se presenta la clasificación de los huracanes de acuerdo con la escala de Saffir – Simpson.\\ 
%En la actualidad se denomina a los huracanes de las categorías 3, 4 y 5 como “huracanes intensos”. Los huracanes de las categorías 4 y 5 pueden denominarse “como los huracanes más intensos” o de “gran intensidad”, los de categorías 2 y 3 cómo intensidad moderada y los de categoría 1 de poca intensidad.


\begin{table}
\begin{center}
\caption{Clasificación de los huracanes según la escala de Saffir – Simpson}

\begin{tabular}{| c | c | c |} 
\hline
Categoría & Viento máximo sostenido (km/h) & Daños\\
\hline\hline
1  & 119 – 153 & Mínimos\\
\hline
2 & 154 – 177 & Moderados\\
\hline
3 & 178 – 208 & Extensos\\
\hline
4 & 209 – 251 & Extremos\\
\hline
5 & >=252 & Catastróficos\\
\hline
\end{tabular}

\label{escala_saffir_simpson}
\end{center}
\end{table}

%El desarrollo de las tecnologías y la implementación de los nuevos modelos numéricos para el monitoreo de los ciclones tropicales ha mejorado significativamente en los últimos años. Esto acompañado de la intensa actividad que implementa la Defensa Civil en Cuba permite reducir el número de víctimas y de daños materiales al país tras el paso de estos sistemas. Es necesario, para la predicción y monitoreo de los huracanes, tener en cuenta su cronología y su comportamiento a lo largo de los años \cite{DK6}.\\

La aplicación de escritorio TkHURS existente será modificada, se le agregarán funcionalidades nuevas. Este software  posee la capacidad de manejo de cronologías, y el análisis de las mismas, agrupando la información en distintas tablas que permiten observar, por ejemplo, la cantidad de huracanes por regiones y calcular el período de retorno de los mismos. Pero por la necesidad de las salidas en forma de gráficos se decidió hacer una actualización que se llamará \textbf{TkHURSv1.1}. Se requiere, para la predicción y monitoreo de los huracanes, tener en cuenta su cronología y su comportamiento a lo largo de los años \cite{DK6}, por lo que el software es muy útil.

\section*{Objetivos}

\subsection*{General}
Realizar la actualización del software TkHurs para el análisis de períodos de retorno a través de un Modelo de Poisson utilizando la cronología de Ciclones Tropicales en Cuba.

\subsection*{Específicos}
\begin{itemize}
\item Calcular las frecuencias estimadas
\item Desarrollar una metodología para el análisis de los de los Ciclones Tropicales. 
\item Realizar la validación a través de la Cronología de los Ciclones Tropicales.
\end{itemize}


El documento cuenta con introducción, primer capítulo que explica brevemente los métodos estadísticos utilizados. En el segundo capítulo se realiza un análisis de la metodología usada y se describe el programa y las herramientas empleadas para su implementación. En el tercer capítulo se muestran y discuten los resultados obtenidos de la utilización del  programa a través de un ejemplo usando la cronología de huracanes. Finalmente se presentan las conclusiones de la investigación, recomendaciones para trabajos futuros y por último la Bibliografía empleada.


\end{introduction}
