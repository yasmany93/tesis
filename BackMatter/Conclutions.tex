\begin{conclusions}

Se realizó la actualización del software (\textbf{TkHURSv1.1}), este resulta ser una herramienta muy útil para las investigaciones y los servicios con beneficios económicos  del Instituto de Meteorología y los Centros Meteorológicos Provinciales del país. Se observó que el modelo de Poisson se ajusta con el nivel de significación requerido a la variable contadora del número de huracanes por año para Cuba, este comportamiento anual está relacionado con un fenómeno de baja frecuencia de ocurrencia. Mediante la fórmula de recurrencia, se observa que las frecuencias esperadas son similares a las observadas obteniéndose buenos resultados. Los valores esperados de los períodos de retorno son resultados coherentes con la realidad.
%El software (\textbf{TkHURSv1.1}) resulta ser una herramienta muy útil para el \textit{INSMET} a la hora de hacer investigaciones y para realizar servicios con beneficios económicos. Se observa que el modelo de Poisson se ajusta con el nivel de significación requerido a la variable contadora del número de huracanes por año para Cuba. Este comportamiento anual está relacionado con un fenómeno de baja frecuencia de ocurrencia.
%Mediante la fórmula de recurrencia, se obtienen frecuencias esperadas que son similares a las observadas obteniéndose buenos resultados. Los valores esperados de los períodos de retorno son resultados coherentes con la realidad.\\


Se actualizó la metodología permitiendo agregar las nuevas funciones del software para una mayor comprensión a la hora de su utilización.\\  
Se realizó la validación a través de la Cronología de los Ciclones Tropicales de la region Central en el período 1980-2018 con los huracanes intensos, se pudo observar que se puede esperar un huracán intenso cada 10 años.
El gráfico de pastel muestra los resultados respecto a las distintas categorías: SS3, SS4 y SS5, observándose que la categoría que más ha afectado esta región es la SS4 con un 50\%.
%También se creó una metodología que permite, mediante el uso de esta aplicación de escritorio, estudiar cronologías de una forma sencilla y de fácil manejo para los usuarios. En la Figura \ref{fig:esquema_metodologia}) se puede observar el esquema creado a partir de dicha metodología y por consiguiente la manera en la que está pensado el software para su uso. Además, en el Capítulo 3 se aprecia la validez de \textbf{TkHURSv1.1}, en el que se explicó a partir de un ejemplo práctico los beneficios que trae.\\

Se implementó una pestaña de “Visualización” de gráficos de barras, pastel y línea para mostrar los resultados de la investigación.
%A parte de agregar la pestaña ``Visualización'' que da paso a las opciones de gráficos, también es posible llevar el software a pantalla completa (otra de las funcionalidad agregada). Aunque hubieron objetivos que no se pudieron alcanzar debido a la situación actual de cuarentena que provocó falta de recursos de investigación, este es el caso de agregar un mapa a la aplicación sin necesidad de conectividad a Internet.



%Los meses de mayor frecuencia de ocurrencia de huracanes son: octubre (37.1\%), septiembre (31.0\%) y agosto (15.5\%) que comprenden el 83.6\%. Los huracanes intensos (SS3, SS4 y SS5) representan el 29.3\% y ocurren con mayor frecuencia en octubre con el 14.7\%. En Cuba, en los últimos 18 años, el 42.9\% de los huracanes de categoría han sido SS4, el 83.3\% huracanes son intensos y el 58.3\% de gran intensidad.

\end{conclusions}

\begin{recomendations}

Implementar en la próxima versión del software en la opción Visualización, el Modo Mapa para obtener salidas que ilustren los resultados de la investigación.\\
Utilizar el software en todos los Centros Meteorológicos Provinciales del país con el objetivo de realizar mejores investigaciones y brindar servicios.
%Como recomendaciones se tiene terminar con el objetivo que no se pudo lograr: integrar un mapa al software. 

\end{recomendations}